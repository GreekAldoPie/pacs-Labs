\documentclass[10pt,aspectratio=169]{beamer}
\usetheme{default}
\setbeamercovered{invisible}
\setbeamertemplate{navigation symbols}{}
\setbeamertemplate{footline}{
    \flushright{\hfill \insertframenumber{}/\inserttotalframenumber}
}

\usepackage{listings}

% User-defined colors.
\definecolor{DarkGreen}{rgb}{0, .5, 0}
\definecolor{DarkBlue}{rgb}{0, 0, .5}
\definecolor{DarkRed}{rgb}{.5, 0, 0}
\definecolor{LightGray}{rgb}{.95, .95, .95}
\definecolor{White}{rgb}{1.0,1.0,1.0}
\definecolor{darkblue}{rgb}{0,0,0.9}
\definecolor{darkred}{rgb}{0.8,0,0}
\definecolor{darkgreen}{rgb}{0.0,0.85,0}

% Settings for listing class.
\lstset{
  language=C++,                        % The default language
  basicstyle=\small\ttfamily,          % The basic style
  backgroundcolor=\color{White},       % Set listing background
  keywordstyle=\color{DarkBlue}\bfseries, % Set keyword style
  commentstyle=\color{DarkGreen}\itshape, % Set comment style
  stringstyle=\color{DarkRed}, % Set string constant style
  extendedchars=true % Allow extended characters
  breaklines=true,
  basewidth={0.5em,0.4em},
  fontadjust=true,
  linewidth=\textwidth,
  breakatwhitespace=true,
  showstringspaces=false,
  lineskip=0ex, %  frame=single
}

\begin{document}
    \title{Plugins \protect\\ Expression templates \protect\\ Generic programming}
    \author{Matteo Caldana}
    \date{18/05/2023}

\begin{frame}[plain, noframenumbering]
    \maketitle
\end{frame}

\begin{frame}[fragile]{Exercise 1: quadrature rules with plugins}
    Let us consider the following simple example compiled with \texttt{g++ 01-main.cpp -o main}
    \lstinputlisting{./examples/01-main.cpp}
    \vfill
    We can use the \texttt{nm} command to show information about symbols in the executable we obtained (it works also with object files, and object-file libraries). In particular, by running the command \texttt{nm -a main | grep func} we will obtain something like:
    
\begin{verbatim}
00000000004010f6 T _Z4funcv
\end{verbatim}

In order to support overloading, the C++ compiler has mangled the name of the function \texttt{func} with the type of the input parameters (v for \texttt{void}).

\end{frame}

\begin{frame}[fragile]{Exercise 1: quadrature rules with plugins}
    This is more evident when we consider an example with function overloading
    \lstinputlisting{./examples/02-main.cpp}
    \vfill
    By using again the \texttt{nm} we will obtain something like:
    
\begin{verbatim}
00000000004010fd T _Z4funci
0000000000401109 T _Z4funcid
00000000004010f6 T _Z4funcv
\end{verbatim}

We have three different symbols, one for each overload. However, notice how only the parameter types (and not the return type) affect the name mangling. Indeed, you can't overload methods based on return type.

\end{frame}
    
\begin{frame}[fragile]{Exercise 1: quadrature rules with plugins}
    What happens in C where there is no function overload? We can test it by using the \texttt{extern "C"} directive.
    \lstinputlisting{./examples/03-main.cpp}
    \vfill
	Now, the symbols defined in the executable concerting the functions are 
\begin{verbatim}
00000000004010f6 T _Z4funcv
00000000004010fd T another_func
\end{verbatim}
Notice how \texttt{another\_func} has no name mangling. This feature will come in handy when dynamically loading plug-ins since we do not want to read a mangled name from the shared library.
\end{frame}


    

\begin{frame}{Exercise 1: quadrature rules with plugins}
\begin{itemize}
\item Implement a code that enables the user to compute the integral of a scalar function by selecting the quadrature rule as the name of a dynamically loadable object;
\item The dynamically loadable object should define a function with the following signature:
\texttt{double integrate(const std::function<double (double)> \&, double a, double b)}.
\item Implement plugins for midpoint, trapezoidal and Simpson's rule.
\item Implement a plugin for quadrature with adaptive refinement.
\end{itemize}
\end{frame}

\begin{frame}{Exercise 2: expression templates}
This exercise is inspired by \url{https://www.modernescpp.com/index.php/expression-templates}.
\vfill
The starting code provided implements a custom \texttt{MyVector} class.
\begin{itemize}
\item Given two vectors $\mathbf{x}$ and $\mathbf{y}$, implement the sum and the multiplication operators to compute the quantity $2\mathbf{x} + \mathbf{y}^2$, where the operations are intended component-wise.
\item Provide an expression template implementation and compare the computational costs with respect to a non-template implementation.
\end{itemize}
\vfill
{\small See also: \url{https://devtut.github.io/cpp/expression-templates.html} and \url{https://en.wikipedia.org/wiki/Expression_templates}.}
\end{frame}

\begin{frame}{Exercise 3: 1D interpolator using generic programming}
This exercise (inspired by \texttt{Examples/src/Interp1D}) shows an example of a generic piecewise linear 1D interpolator, with an application to
vectors of couples of values (representing the interpolation nodes and the corresponding values) and to two separate vectors (one for the nodes and one for the values).

It is rather generic and accepts as an input iterators to any container with bi-directional iterators. Bi-directionality is indeed only needed for extrapolation on the right.
\end{frame}
\end{document}
